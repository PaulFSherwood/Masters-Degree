% Preamble is what is before the \begin document line
\documentclass{report}%article} %report}
\usepackage[utf8]{inputenc}
\usepackage{mathtools, amsmath, amssymb, amsthm}
% \input{preamble}
% \input{macros}
% \input{letterfonts}

\title{\Huge{LaTex for Students}}
\author{\huge{Paul Sherwood}}
\date{March 17, 2023}

\begin{document}
\maketitle
% change the starting chapter number
\setcounter{chapter}{2}
\chapter{\centering Introduction}
% \chapter*{\centering Preface}
\section{backgrounds}
\section{will be covered}
\section{Introduction}\label{sec:intro}

LaTex is a \textbf{software} system \textit{for document} preperastion, \underline{in a way a sort} of markup or programming language.  It produces beautify looking documents in a professional manner.  Students can find uses for LaTeX everywhere-while the obvious place are in math and science classes, it can be used anywhere.
\sectionmark{code}
lajsdf

\subsection{Why use LaTeX?}

\section{Getting Started}
ipsum dolor sit amet, consectetur adipiscing elit. Nulla vitae nunc nec nisl lacinia lacinia. Nulla


\[
% infinity
%    E
%  n - 1
\sum_{n = 1}^{\infty}
\frac{1}{n^2} = \frac{\pi^2}{\phi }
\]

\begin{equation}\label{eq:sample01}
    \mathbb{N}
    \Rightarrow
    \lim_{x \to \mathfrak{n}} 
    \Leftarrow
    \mathbb{Z}
\end{equation}

\[
B = \underbrace{\{\{\{ . . . \}\}\}}_{\text{B}}
\]

In 1902, Einstein created this equation: $E=mc^2$.  This equation is the basis of modern physics.

$\sum F=ma$

\begin{equation} \label{eq1}
    \begin{split}
        A & = \frac{\pi r^2}{2} \\
        A & = \frac{1}{2} \pi r^2
    \end{split}
\end{equation}

\ref{sec:intro} as shown in \ref{eq:sample01}

\end{document}